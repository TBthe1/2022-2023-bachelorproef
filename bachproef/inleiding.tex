%%=============================================================================
%% Inleiding
%%=============================================================================

\chapter{\IfLanguageName{dutch}{Inleiding}{Introduction}}%
\label{ch:inleiding}

%De inleiding moet de lezer net genoeg informatie verschaffen om het onderwerp te begrijpen en in te zien waarom de onderzoeksvraag de moeite waard is om te onderzoeken. In de inleiding ga je literatuurverwijzingen beperken, zodat de tekst vlot leesbaar blijft. Je kan de inleiding verder onderverdelen in secties als dit de tekst verduidelijkt. Zaken die aan bod kunnen komen in de inleiding~\autocite{Pollefliet2011}:
%
%\begin{itemize}
%  \item context, achtergrond
%  \item afbakenen van het onderwerp
%  \item verantwoording van het onderwerp, methodologie
%  \item probleemstelling
%  \item onderzoeksdoelstelling
%  \item onderzoeksvraag
%  \item \ldots
%\end{itemize}

Een webshop is een website waarop goederen en/of diensten worden verkocht. Om een webshop van inhoud te voorzien moet deze zo eenvoudig mogelijk opgevuld worden. Hiervoor bestaan er verschillende tools, waaronder een CMS of contentmanagementsysteem. Hiermee kan met behulp van een inlogsysteem de klant de webshop bewerken, zonder enige programmeerkennis. De co-promotor gebruikt hiervoor WordPress. Standaard is WordPress maar een website voor blogberichten. Via uitbreidingen op deze tool, bekend onder de naam plugins, kan er zo een webshop worden gemaakt. Een bekende plugin die de co-promotor hiervoor gebruikt is WooCommerce. Merk op dat er nog veel andere CMS'en en WordPress plugins bestaan, maar deze niet tot in detail worden besproken in deze paper, aangezien de co-promotor met deze tools zal blijven verder werken. 
\\\\
Het overnemen van een webshop betekent dat je een nieuwe webshop maakt, en zoveel mogelijk van de originele webshop tracht over te nemen. Idealiter worden problemen of mankementen bij deze overname niet meegenomen. Voor de co-promotor betekent dit een nieuwe WordPress met WooCommerce opzetten, en zoveel mogelijk trachten over te nemen van de originele webshop.

\section{\IfLanguageName{dutch}{Probleemstelling}{Problem Statement}}%
\label{sec:probleemstelling}

%Uit je probleemstelling moet duidelijk zijn dat je onderzoek een meerwaarde heeft voor een concrete doelgroep. De doelgroep moet goed gedefinieerd en afgelijnd zijn. Doelgroepen als ``bedrijven,'' ``KMO's'', systeembeheerders, enz.~zijn nog te vaag. Als je een lijstje kan maken van de personen/organisaties die een meerwaarde zullen vinden in deze bachelorproef (dit is eigenlijk je steekproefkader), dan is dat een indicatie dat de doelgroep goed gedefinieerd is. Dit kan een enkel bedrijf zijn of zelfs één persoon (je co-promotor/opdrachtgever).

Team Made, en ook andere bedrijven die aan webshops overnames doen, verliest veel tijd aan zo een overname. Het is ook niet altijd even voor de hand liggend hoeveel tijd en budgettering zo een overname precies in beslag zal nemen. Daarnaast moet de webshop ook op een veilige manier worden opgebouwd, die liefst zo lang mogelijk mee gaat in deze zeer snel groeiende technologische wereld. We willen niet dat een splik splinternieuwe webshop na een paar weken al out-of-date is. We moeten ervoor zorgen dat de code zo future-proof mogelijk is.

\section{\IfLanguageName{dutch}{Onderzoeksvraag}{Research question}}%
\label{sec:onderzoeksvraag}

%Wees zo concreet mogelijk bij het formuleren van je onderzoeksvraag. Een onderzoeksvraag is trouwens iets waar nog niemand op dit moment een antwoord heeft (voor zover je kan nagaan). Het opzoeken van bestaande informatie (bv. ``welke tools bestaan er voor deze toepassing?'') is dus geen onderzoeksvraag. Je kan de onderzoeksvraag verder specifiëren in deelvragen. Bv.~als je onderzoek gaat over performantiemetingen, dan 

Zijn er vandaag de dag genoeg tools aanwezig die dit overnameproces kunnen stroomlijnen? Is er een manier om dit ook voor meerdere klanten tegelijkertijd sneller te laten verlopen? Kan alles met behulp van één tool worden versneld, of moet een combinatie van tools worden gebruikt? Het zijn allemaal vragen die we ons moeten stellen om het overnameproces te willen stroomlijnen.

\section{\IfLanguageName{dutch}{Onderzoeksdoelstelling}{Research objective}}%
\label{sec:onderzoeksdoelstelling}

%Wat is het beoogde resultaat van je bachelorproef? Wat zijn de criteria voor succes? Beschrijf die zo concreet mogelijk. Gaat het bv.\ om een proof-of-concept, een prototype, een verslag met aanbevelingen, een vergelijkende studie, enz.

Er zal onderzoek worden gedaan naar welke stappen noodzakelijk zijn voor een webshop overname, en hoe deze vereenvoudigd of zelfs geautomatiseerd kunnen worden. Hierbij zullen we kort analyseren of de tools die Team Made gebruikt, namelijk WordPress en WooCommerce, betere alternatieven hebben of niet. We wensen dat Team Made een (combinatie van) (zelfgeschreven) tool(s) kan gebruiken om zijn bestaande workflow te optimaliseren. Als een overname sneller gaat, of goedkopere alternatieven te bieden heeft die niet inboeten op tijd en kwaliteit, dan zitten we op het juiste spoor.

\section{\IfLanguageName{dutch}{Opzet van deze bachelorproef}{Structure of this bachelor thesis}}%
\label{sec:opzet-bachelorproef}

% Het is gebruikelijk aan het einde van de inleiding een overzicht te
% geven van de opbouw van de rest van de tekst. Deze sectie bevat al een aanzet
% die je kan aanvullen/aanpassen in functie van je eigen tekst.

De rest van deze bachelorproef is als volgt opgebouwd:

In Hoofdstuk~\ref{ch:stand-van-zaken} wordt een overzicht gegeven van de stand van zaken binnen het onderzoeksdomein, op basis van een literatuurstudie.

In Hoofdstuk~\ref{ch:methodologie} wordt de methodologie toegelicht en worden de gebruikte onderzoekstechnieken besproken om een antwoord te kunnen formuleren op de onderzoeksvragen.

In Hoofdstuk~\ref{ch:stappenplan} wordt een kort overzicht gegeven van de stappen die men moet ondernemen voor het maken en overnemen van een webshop.

In Hoofdstuk~\ref{ch:conclusie}, tenslotte, wordt de conclusie gegeven en een antwoord geformuleerd op de onderzoeksvragen. Daarbij wordt ook een aanzet gegeven voor toekomstig onderzoek binnen dit domein.
