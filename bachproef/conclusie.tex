%%=============================================================================
%% Conclusie
%%=============================================================================

\chapter{Conclusie}%
\label{ch:conclusie}

% TODO: Trek een duidelijke conclusie, in de vorm van een antwoord op de
% onderzoeksvra(a)g(en). Wat was jouw bijdrage aan het onderzoeksdomein en
% hoe biedt dit meerwaarde aan het vakgebied/doelgroep? 
% Reflecteer kritisch over het resultaat. In Engelse teksten wordt deze sectie
% ``Discussion'' genoemd. Had je deze uitkomst verwacht? Zijn er zaken die nog
% niet duidelijk zijn?
% Heeft het onderzoek geleid tot nieuwe vragen die uitnodigen tot verder 
%onderzoek?
\section{Ondervindingen}
Veel aspecten bij een overname van een webshop zijn en blijven manueel werk, maar het is duidelijk dat dit alleen maar zal afnemen. Veel opkomende AI-tools die nog niet publiek toegankelijk zijn hebben in demo's al veelbelovende resultaten opgebracht die in dit proces zullen helpen, zoals de demo van Greg Brockman die op basis van een schets een webpagina kan coderen \autocite{Das2023}.
\\\\
Op moment van schrijven zijn veel mensen zoals Geoffrey Hinton \autocite{ZoeKleinman2023} bezorgd om de snelle groei rond AI en wat de ethische gevolgen van deze groei zullen zijn. Op vlak van GDPR zullen er ongetwijfeld gigantisch veel aanpassingen komen die de regels voor het overnemen van (AI-gegenereerde) code, en privacy rond (AI-gegenereerde) beeldmaterialen zullen bepalen.
\\\\
De auteur heeft tijdens zijn stageperiode persoonlijk ondervonden dat AI-tools niet alles perfect zelfstandig kunnen verwezenlijken. Er is nog steeds veel dat de gebruiker zelf moet voorzien. Wat voor een mens logisch lijkt kan een AI-tool niet weten als die daar niet specifiek op getraind is. Belangrijk is dat men AI-tools gaat gebruiken als een extensie, en niet als een vervanging. Hoe meer informatie je zelf kan meegegeven in een vraagstelling, hoe hoger de slaagkansen van een werkbaar resultaat. Er zijn nu reeds een aantal AI-tools aanwezig die Team Made kunnen assisteren. 
\\\\
\section{Manueel werk}
Volgende zaken moeten op moment van schrijven nog manueel uitgevoerd worden:
\begin{itemize}
    \item Het opvullen van de klantengegevens in een configuratiebestand.
    \item Het ophalen van een logo voor een thema.
    \item Het stijlen van de website.
    \item Het koppelen van de artikelendata hun velden bij de import van de WooCommerce plugin.
\end{itemize}
Daarnaast is het sterk aangeraden om volgende zaken toch nog manueel uit te voeren voor kleinschalige projecten:
\begin{itemize}
    \item Een databank aanmaken (en koppelen).
    \item Het installeren van WordPress o.b.v. de config-file in de browser.
\end{itemize}
WP-CLI is een zeer krachtige tool om alles binnenin WordPress te automatiseren, maar vereist enige programmeerkennis. Het is aangeraden om de manuele installatie van WordPress via de browser op te volgen voor een snelle installatie. Het is sneller om een installatie via deze weg uit te voeren dan een volledig nieuw script te schrijven. Echter indien er veel websites in één keer moeten geïnstalleerd worden zal het schrijven van één script die dit allemaal opvangt veel tijd kunnen besparen. Hier gaat het stappenplan\ref{ch:stappenplan} goed van pas komen. Men moet per functionaliteit zelf goed afwegen of het investeren van tijd in het maken van scripten gaat opwegen tegenover het manueel uit te voeren of niet.
\\\\
Om het stappenplan zo goed mogelijk te kunnen opvolgen is het sterk aangeraden om scriptie per functionaliteit toe te passen, en niet alles in één script te verwerken. Dit helpt ook bij het debuggen door sneller eventuele fouten op te sporen (waarbij niet één bestand volledig moet doorgenomen worden).
\\\\
Het overnemen en stijlen van een website is momenteel nog manueel werk, maar zal niet lang meer duren. In de maand Maart van 2023 is Adobe met een project genaamd Firefly gekomen. Hiermee kunnen gebruikers in Photoshop afbeeldingen en zelfs video's maken m.b.v. AI. De functionaliteit om enkel met tekst realistische beelden te maken wijst naar een nabije toekomst waarbij we op basis van tekst stukken code kunnen genereren. Bij een demo van Elementor hun nieuwe AI-tool (zie hoofdstuk \ref{elementor_ai_hoofdstuk}) konden we dit reeds in beperkte mate aanschouwen. 
\\\\
\section{Nieuwe verwachtingen}
Er zijn visueel veel gelijkenissen in de lijst van Team Made hun webshops. De verwachtingen zijn dat de opkomende AI-tools dit ook zullen opmerken en bijgevolg veel gelijkaardige code zullen genereren. Het is en blijft een compleet nieuwe technologie die zeer onvoorspelbaar kan zijn \autocite{KnorrEvans2023}.
\\\\
De verwachtingen zijn dat er enorm veel AI-tools gaan blijven bijkomen om niet alleen het leven van de programmeurs, maar ook andere jobs te vereenvoudigen. Dat er één AI-tool zal ontstaan die de volledige opzet, opvulling, onderhoud en het online zetten van een webshop op zich kan nemen, lijkt zeer realistisch en niet meer zo veraf te zijn. Er is nog steeds een expert nodig die begrijpt wat er allemaal aan het gebeuren is. Niet alleen om de noodzakelijke aanpassingen uit te voeren, maar ook omdat het concept van automatiseren met AI nog steeds veel manuele input vergt. 
\\\\
Van zodra image-recognition in AI-tools voor iedereen toegankelijk wordt dan gaat het overnemen van een bestaande webshop gigantisch snel gaan. Screenshots kunnen nemen van een design, en vragen om dit in code uit te schrijven, gaat enorm veel tijd besparen. Op moment van schrijven kan dat nog niet en moet de gebruiker zeer gedetailleerd beschrijven wat hij van code wenst te ontvangen.
\section{Aanbevelingen voor de toekomst}
Zoals kort besproken in het hoofdstuk “Eervolle vermeldingen” (zie \ref{eervolle_vermeldingen}) zal Elementor AI (zie \ref{elementor_ai_hoofdstuk}) hoogstwaarschijnlijk een interessante optie zijn voor Team Made, die heel wat zaken hiermee kan automatiseren. Het is sterk aangeraden om hiermee te experimenteren van zodra het product publiek toegankelijk is.
\\\\
Probeer in het algemeen zoveel mogelijk verschillende AI-tools uit, en bekijk welke de beste resultaten specifiek voor een bepaald project opleveren. Dezelfde vraag in tool A kan een volledig ander resultaat opleveren in tool B. Blijf het aanschouwen als een hulpmiddel, en niet als een volledig vervangmiddel.

