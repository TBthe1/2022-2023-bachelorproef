%%=============================================================================
%% Conclusie
%%=============================================================================

\chapter{Conclusie}%
\label{ch:conclusie}

% TODO: Trek een duidelijke conclusie, in de vorm van een antwoord op de
% onderzoeksvra(a)g(en). Wat was jouw bijdrage aan het onderzoeksdomein en
% hoe biedt dit meerwaarde aan het vakgebied/doelgroep? 
% Reflecteer kritisch over het resultaat. In Engelse teksten wordt deze sectie
% ``Discussion'' genoemd. Had je deze uitkomst verwacht? Zijn er zaken die nog
% niet duidelijk zijn?
% Heeft het onderzoek geleid tot nieuwe vragen die uitnodigen tot verder 
%onderzoek?
\section{Switch van aanpak}
Het oorspronkelijke plan was om een tool te schrijven die (deels) stappen bij het proces van een webshop overname kan versnellen. Rekening houdende met de co-promotor mag deze tool niet te complex zijn, en moet naar toekomstig gebruik toe gemakkelijk schaalbaar zijn.
\\\\
Echter tijdens het schrijven van het voorstel voor deze bachelorproef waren AI-tools zoals ChatGPT nog onbestaand. Het is na de literatuurstudie heel erg duidelijk dat met behulp van (een combinatie van) AI-tools de co-promotor zijn doel kan bereiken. De auteur heeft in de loop van het schrijven ondervonden dat veel programmeurs, waaronder hijzelf, dagelijks code laten genereren door ChatGPT. Deze chatbot is doorheen de loop van tijd ook meermaals van updates voorzien en aanzienlijk veel sneller geworden. 
\\\\
De hele opzet van WordPress is geautomatiseerd met behulp van scripting in PowerShell. Het overnemen van bestaande webshops hun designs en data werd verwezenlijk in combinatie van manuele programmeerwerk en AI-tools. 

\section{Ondervindingen}
Veel aspecten bij een overname van een webshop zijn en blijven manueel werk, maar het is duidelijk dat dit alleen maar zal afnemen. Veel opkomende AI-tools die nog niet publiek toegankelijk zijn hebben in demo's al veelbelovende resultaten opgebracht die in dit proces zullen helpen, zoals de demo van Greg Brockman die op basis van een schets een webpagina kan coderen \autocite{Das2023}.
\\\\
Op moment van schrijven zijn veel mensen zoals Geoffrey Hinton \autocite{ZoeKleinman2023} bezorgd om de snelle groei rond AI en wat de ethische gevolgen van deze groei zullen zijn. Op vlak van GDPR zullen er ongetwijfeld gigantisch veel aanpassingen komen die de regels voor het overnemen van (AI-gegenereerde) code, en privacy rond (AI-gegenereerde) beeldmaterialen zullen bepalen.
\\\\
De auteur heeft tijdens zijn stageperiode persoonlijk ondervonden dat AI-tools niet alles perfect zelfstandig kunnen verwezenlijken. Er is nog steeds veel dat de gebruiker zelf moet voorzien. Wat voor een mens logisch lijkt kan een AI-tool niet weten als die daar niet specifiek op getraind is. Belangrijk is dat men AI-tools gaat gebruiken als een extensie, en niet als een vervanging. Hoe meer informatie je zelf kan meegegeven in een vraagstelling, hoe hoger de slaagkansen van een werkbaar resultaat. Er zijn nu reeds een aantal AI-tools aanwezig die de co-promotor kunnen assisteren. 
\\\\
\section{Nieuwe verwachtingen}
Er zijn visueel veel gelijkenissen in de lijst van de co-promotor zijn webshops. De verwachtingen zijn dat de AI-tools dit ook zullen opmerken en bijgevolg veel gelijkaardige code zullen genereren. Het is en blijft een compleet nieuwe technologie die zeer onvoorspelbaar kan zijn \autocite{KnorrEvans2023}.
\\\\
De verwachtingen zijn dat er enorm veel AI-tools gaan blijven bijkomen om niet alleen het leven van de programmeurs, maar ook andere jobs te vereenvoudigen. Dat er één AI-tool zal ontstaan die de volledige opzet, opvulling, onderhoud en het online zetten van een webshop op zich kan nemen, lijkt zeer realistisch en niet meer zo ver af te zijn. Er is nog steeds een expert nodig die begrijpt wat er allemaal aan het gebeuren is. Niet alleen om de noodzakelijke aanpassingen uit te voeren, maar ook omdat het concept van automatiseren met AI nog steeds veel manuele input vergt. 
\\\\
Van zodra image-recognition in AI-tools voor iedereen toegankelijk wordt dan gaat het overnemen van een bestaande webshop gigantisch snel gaan. Het kunnen screenshotten van een design, en vragen om dit in code uit te schrijven, gaat enorm veel tijd besparen. Op moment van schrijven kan dat nog niet en moet de gebruiker zeer gedetailleerd beschrijven wat hij van code wenst te ontvangen.
\section{Aanbevelingen voor co-promotor}
Zoals kort besproken in het hoofdstuk “Eervolle vermeldingen” zal Elementor AI hoogstwaarschijnlijk een interessante optie zijn voor de co-promotor, die heel wat zaken hiermee kan automatiseren. Probeer zoveel mogelijk verschillende AI-tools uit, en bekijk welke de beste resultaten specifiek voor een bepaald project opleveren. Dezelfde vraag in tool A kan een volledig ander resultaat opleveren in tool B. 

