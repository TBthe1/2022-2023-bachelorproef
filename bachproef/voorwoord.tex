%%=============================================================================
%% Voorwoord
%%=============================================================================

\chapter*{\IfLanguageName{dutch}{Woord vooraf}{Preface}}%
\label{ch:voorwoord}

%% TODO:
%% Het voorwoord is het enige deel van de bachelorproef waar je vanuit je
%% eigen standpunt (``ik-vorm'') mag schrijven. Je kan hier bv. motiveren
%% waarom jij het onderwerp wil bespreken.
%% Vergeet ook niet te bedanken wie je geholpen/gesteund/... heeft
Graag wil ik van deze gelegenheid gebruikmaken om mijn promotor S. Lambert te bedanken voor de talloze feedback momenten, en om mijn co-promotor D. Matthijs te bedanken voor het aanbieden van deze interessante bedrijfscasus. Als webontwikkelaar die al enkele webshops gemaakt heeft, sprak deze uitdaging mij direct aan.
\\\\
Afgelopen periode van technologische vooruitgang was, voornamelijk op AI-vlak, een heel erg leerzaam en boeiende ervaring. Desondanks de vele literatuurstudies die ongetwijfeld heel snel zullen achterhaald worden, bleef het fascinerend om te zien hoe alles enorm snel aan het evolueren is. Het is waar dat we op technologisch vlak snel veranderingen meemaken, maar een AI-evolutie zoals deze zagen we allemaal niet aankomen. Ondertussen heeft bijna heel de wereld over ChatGPT gehoord, of al eens gebruikt. Nooit had ik mij kunnen voorstellen dat gedurende heel mijn stageperiode een persoonlijke assistent mij kon assisteren bij het programmeren. Afgelopen maanden was er elke dag iets nieuws over AI te vertellen en het brengt tot op vandaag nog altijd veel media-aandacht met zich mee. Bedrijven zijn volop aan de gang met bestaande tools van AI te voorzien om ons leven eenvoudiger te maken. Binnenkort hoeven we zelfs onze eigen e-mails niet meer te schrijven.
\\\\
Als iemand geboren in de jaren '90 en met een grote passie voor technologie is dit echt adembenemend om mee te mogen maken. Het is aan de ene kant alarmerend maar tegelijkertijd ook fascinerend hoeveel werk voor ons op zoveel vlakken geautomatiseerd kan worden. Bij aanvang van deze paper was het niet de bedoeling om met AI te werken, maar het is heel erg duidelijk geworden dat AI hierbij het meeste kan helpen.
\\\\
Op moment van schrijven is het nog niet helemaal duidelijk hoever we staan in deze evolutie maar één ding is zeker: AI is de toekomst.