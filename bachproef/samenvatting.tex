%%=============================================================================
%% Samenvatting
%%=============================================================================

% TODO: De "abstract" of samenvatting is een kernachtige (~ 1 blz. voor een
% thesis) synthese van het document.
%
% Een goede abstract biedt een kernachtig antwoord op volgende vragen:
%
% 1. Waarover gaat de bachelorproef?
% 2. Waarom heb je er over geschreven?
% 3. Hoe heb je het onderzoek uitgevoerd?
% 4. Wat waren de resultaten? Wat blijkt uit je onderzoek?
% 5. Wat betekenen je resultaten? Wat is de relevantie voor het werkveld?
%
% Daarom bestaat een abstract uit volgende componenten:
%
% - inleiding + kaderen thema
% - probleemstelling
% - (centrale) onderzoeksvraag
% - onderzoeksdoelstelling
% - methodologie
% - resultaten (beperk tot de belangrijkste, relevant voor de onderzoeksvraag)
% - conclusies, aanbevelingen, beperkingen
%
% LET OP! Een samenvatting is GEEN voorwoord!

%%---------- Nederlandse samenvatting -----------------------------------------

\IfLanguageName{english}{%
\selectlanguage{dutch}
\chapter*{Samenvatting}
\selectlanguage{english}
}{}

%%---------- Samenvatting -----------------------------------------------------
% De samenvatting in de hoofdtaal van het document

\chapter*{\IfLanguageName{dutch}{Samenvatting}{Abstract}}

De co-promotor D.Matthijs heeft een bedrijf Team Made dat niet alleen websites en webshops bouwt, maar ook overneemt. Het proces van een webshop over te nemen, en om te zetten naar een WordPress met WooCommerce, kan veel tijd in beslag nemen. Deze paper doet een onderzoek en houdt een proof-of-concept om dit proces te doen stroomlijnen.
\\\\
Een webshop overnemen is niet zomaar evident en vereist enige programmeerkennis. Er kunnen veel (externe) factoren zorgen voor een complexe overdracht. Klanten moeten hun eigen webshop zelf kunnen beheren, een webshop kan een zeer complexe lay-out hebben, er kunnen reeds veel artikelen aanwezig zijn enzovoort. Hoe kunnen we een overname zo eenvoudig en minst tijdrovend mogelijk maken?
\\\\
Eerst en vooral gaan we kijken naar de huidige middelen op de markt die ons kunnen helpen bij deze problematiek. Vervolgens proberen we met een combinatie van bestaande en eventueel zelf ontwikkelde tools tot een resultaat te komen dat de co-promotor tijd doet winnen. En dat allemaal zonder onder te doen aan kwaliteit. 
\\\\
Het resultaat van deze paper is een volledig geautomatiseerd proces voor de hele opzet en installatie, met zo weinig mogelijk manueel werk voor de co-promotor. Het voorzien van de lay-out en het opvullen van de data, kan met behulp van een groeiend aantal AI-tools veel sneller worden verwezenlijkt.
\\\\
Het is sterk aanbevolen om zoveel mogelijk te laten automatiseren met de tools die WordPress ons aanbiedt. Het overnemen van een webshop blijft echter een proces dat enig manueel werk vergt. Er zal altijd een expert nodig zijn voor het beste resultaat te bekomen. De huidig lopende AI-evolutie biedt op moment van schrijven ons talloze opties aan die alleen maar met de dag beter worden, maar ze zijn niet de eindoplossing.


%KOMEN LATER UITGEBREIDER AAN BOD
%
%Inleiding + kaderen thema:
%\begin{itemize}
%    \item Zeggen wat een CMS is, en wat bijgevolg WordPress is
%    \item Zeggen wat een webshop is, en uitleggen wat WooCommerce is
%    \item Zeggen dat er nog andere opties op de markt bestaan (zowel cms als webshop plugins)
%    \item Uitleg over wat een webshop overname precies is
%\end{itemize}
%
%Probleemstelling
%\begin{itemize}
%    \item Veel opties op de markt (zowel CMS als plugins)
%    \item Momenteel veel manueel werk (voor co-promotor) bij een webshop overname
%    \item Doelgroep = mijn co-promotor zijn bedrijf, maar ook andere IT-bedrijven die webshops overnemen?
%    \item Veiligheid bespreken
%    \item Schendig van rechten bij overname bespreken (GDPR)
%    \item Future-proof tool (bv. code dat breekt bij updates)
%\end{itemize}
%
%(centrale) onderzoeksvraag
%\begin{itemize}
%    \item --> hoe kan dit proces eenvoudiger?
%    \item Vooral nu erg belangrijk om te zien of AI (deels) een oplossing kan bieden
%\end{itemize}
%
%onderzoeksdoelstelling:
%\begin{itemize}
%    \item Kijken welke stappen eenvoudiger / geautomatiseerd kunnen
%    \item Co-promotor wilt WP en WC gebruiken, wel ok? Geen betere alternatieven? Kort analyseren waarom
%    \item Bestaat er reeds een (AI) tool die dit (deels) kan?
%    \item Indien ja, kan mijn co-promotor dit gebruiken? Zijn er limieten?
%    \item Indien neen, kan een tool geschreven worden die dit (deels) kan?
%    \item Beoogde resultaat: een (zelfgeschreven) tool die het gemakkelijker maakt / tijd bespaart om webshops over te nemen
%    \item Criteria voor succes: overname gaat sneller, bespaart kosten (bv. met goedkopere plugins)
%\end{itemize}
%
%methodologie:
%\begin{itemize}
%    \item Bestaande (AI) tools overlopen + testen op hun limieten
%    \item Kijken of een tool alles kan, of een combo van meerdere nodig is
%\end{itemize}
%
%resultaten (beperk tot de belangrijkste, relevant voor de onderzoeksvraag):
%\begin{itemize}
%    \item kan ik pas na de literatuurstudie neerschrijven?
%    \item komen hier mijn verwachtingen te staan?
%\end{itemize}
%
%conclusies, aanbevelingen, beperkingen
%\begin{itemize}
%    \item kan ik pas op het einde schrijven?
%\end{itemize}