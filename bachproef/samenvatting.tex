%%=============================================================================
%% Samenvatting
%%=============================================================================

% TODO: De "abstract" of samenvatting is een kernachtige (~ 1 blz. voor een
% thesis) synthese van het document.
%
% Een goede abstract biedt een kernachtig antwoord op volgende vragen:
%
% 1. Waarover gaat de bachelorproef?
% 2. Waarom heb je er over geschreven?
% 3. Hoe heb je het onderzoek uitgevoerd?
% 4. Wat waren de resultaten? Wat blijkt uit je onderzoek?
% 5. Wat betekenen je resultaten? Wat is de relevantie voor het werkveld?
%
% Daarom bestaat een abstract uit volgende componenten:
%
% - inleiding + kaderen thema
% - probleemstelling
% - (centrale) onderzoeksvraag
% - onderzoeksdoelstelling
% - methodologie
% - resultaten (beperk tot de belangrijkste, relevant voor de onderzoeksvraag)
% - conclusies, aanbevelingen, beperkingen
%
% LET OP! Een samenvatting is GEEN voorwoord!

%%---------- Nederlandse samenvatting -----------------------------------------

\IfLanguageName{english}{%
\selectlanguage{dutch}
\chapter*{Samenvatting}
\selectlanguage{english}
}{}

%%---------- Samenvatting -----------------------------------------------------
% De samenvatting in de hoofdtaal van het document

\chapter*{\IfLanguageName{dutch}{Samenvatting}{Abstract}}

Dit onderzoek kwam tot stand naar aanleiding van een vraag van het bedrijf Team Made, dat zelf websites en webshops bouwt, maar deze vaak ook overneemt. Het proces van een webshop over te nemen, en om te zetten naar een WordPress met WooCommerce, kan veel tijd in beslag nemen. Deze paper doet een onderzoek en houdt een proof-of-concept om dit proces te doen stroomlijnen.
\\\\
Een webshop overnemen vereist enige programmeerkennis. Er kunnen veel (externe) factoren zorgen voor een moeilijke overdracht. Klanten moeten hun eigen webshop zelf kunnen beheren, een webshop kan een ingewikkelde lay-out hebben, er kunnen al veel artikelen aanwezig zijn enzovoort. Hoe kunnen we een overname zo eenvoudig en minst tijdrovend mogelijk maken?
\\\\
Eerst en vooral gaan we kijken naar de huidige middelen op de markt die ons kunnen helpen bij deze problematiek. Vervolgens proberen we met een combinatie van bestaande en eventueel zelf ontwikkelde tools tot een resultaat te komen dat Team Made tijd doet winnen. En dat allemaal zonder onder te doen aan kwaliteit. 
\\\\
Het resultaat van deze paper is een volledig geautomatiseerd proces voor de hele opzet en installatie, met zo weinig mogelijk manueel werk voor Team Made. Het voorzien van de lay-out en het opvullen van de data, kan met behulp van een groeiend aantal AI-tools veel sneller worden verwezenlijkt.
\\\\
Het is sterk aanbevolen om zo veel mogelijk te laten automatiseren met de tools die WordPress ons aanbiedt. Het overnemen van een webshop blijft toch een proces dat enig manueel werk vergt. Er zal altijd een expert nodig zijn voor het beste resultaat te krijgen. De huidig lopende AI-evolutie biedt op moment van schrijven ons talloze opties aan die alleen maar met de dag beter worden, maar ze zijn niet de eindoplossing.